\tableofcontents

\chapter{Introdução}
  
Os sistemas complexos e distribuídos são frequentemente utilizados no meio corporativo e industrial tanto para solucionar problemas, que antes não era possível encontrar uma solução ideal quanto adotar novas abordagens para problemas já conhecidos e solucionados. Entre estas abordagens encontram-se os sistemas multiagentes \cite{benfield2006making}.

Os agentes possuem propriedades como autonomia, reatividade, proatividade e habilidades sociais \cite{jennings2000agent}. Segundo \citet{hubner2007developing} em uma sociedade de agentes esta autonomia pode levar ao comportamento indesejado do sistema. Para resolver este tipo de comportamento os SMA podem ser representados como uma organização através de modelos organizacionais. Estes modelos coordenam os agentes em grupos e hierarquias fazendo com que eles sigam regras comportamentais específicas \cite{van2005formal, argente2006multi}.

Mesmo possuindo estruturas organizadas os SMA ainda possuem poucas garantias de que seu funcionamento é correto, sendo este um dos obstáculos para uma maior adoção desta abordagem na indústria \cite{houhamdi2011multi,winikoff2010assurance}. Um modo de obter essa garantia é através de Teste de Software. Os testes ajudam a medir a qualidade do software em termos de números de defeitos encontrados \cite{graham2008foundations}. No entanto, sabe-se que testar SMA não é uma tarefa trivial \cite{winikoff2010assurance}.


\section{Objetivos}

O objetivo geral deste trabalho é desenvolver uma metodologia para avaliar a testabilidade de sistemas multiagentes levando em consideração a dimensão da organização destes sistemas, utilizando como base o teste de caixa-branca e a Rede de Petri como ferramenta para modelagem e análise dos resultados.

Os objetivos específicos a seguir servem como base para o desenvolvimento da pesquisa:

\begin{itemize}

\item Estudar metodologias e modelos para organização de SMA;
\item Estudar abordagens de teste de software encontrados na engenharia de software tradicional;
\item Desenvolver uma metodologia para avaliar a testabilidade em SMA no nível de organização;
\item Validar Rede de Petri como a ferramenta a ser utilizada na modelagem de SMA com nível de organização;
\item Desenvolver uma ferramenta para o cálculo de caminhos de uma rede de petri.
\item Avaliar a metodologia e a utilização da ferramenta nos testes em SMA no nível de organização.
\item Avaliar a testabilidade de SMA que utilizam $\mathcal{M}$oise$^{+}$ como modelo organizacional.

\item Avaliar 


\end{itemize}


\section{Organização do Trabalho}

O trabalho está organizado da seguinte forma:

\begin{itemize}
\item O capítulo 2 apresenta uma revisão bibliográfica relevante para o entendimento do trabalho com temas como: Teste de Software, sendo a primeira parte relativa aos testes na Engenharia de Software tradicional e a segunda parte uma visão dos testes em SMA. Inteligência artificial e sistemas multiagentes. Uma sessão dedicada à engenharia de software orientada a agentes e a última sessão apresenta conceitos de Redes de Petri.

\item No capítulo 3 é feita uma revisão bibliográfica de trabalhos relacionados ao teste em SMA em diferentes níveis, metodologias e comparados com a proposta deste trabalho.

\item No capítulo 4 a metodologia para avaliação da testabilidade do modelo organizacional $\mathcal{M}$oise$^{+}$ baseada em Redes de Petri é apresentada.

\item No capítulo 5 são utilizados exemplos para implementar a metodologia exposta no capítulo anterior.

\item O capítulo 6 apresenta as conclusões do trabalho, suas principais contribuições assim como suas limitação, que expõem oportunidades para trabalhos futuros.
\end{itemize}

