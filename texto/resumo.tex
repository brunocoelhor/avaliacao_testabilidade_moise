%Resumo em Portugues (no maximo 500 palavras)
\begin{abstract}
Os sistemas baseados em Inteligência Artificial Distribuída (IAD) são utilizados para solucionar problemas computacionais onde os sistemas tradicionais não são capazes de solucionar. Os Sistemas Multiagentes (SMA) fazem parte de um dos ramos da IAD, e além das propriedades comuns da IAD como ser um sistema assíncrono e distribuído, os agentes ainda possuem propriedades como autonomia, reatividade, pro-atividade e habilidade social, o que torna muito difícil prever completamente os comportamentos nestes sistemas. 

Para limitar o comportamento em SMA, modelos organizacionais, como o $\mathcal{M}$oise$^{+}$, podem ser empregados para especificar o sistema a partir de um modelo de organização. Estes modelos  estruturam os agentes em grupos, onde os membros destes grupos possuem papéis a desempenhar e também restrições a obedecer. 

Mesmo com este nível de controle sobre os SMA, comportamentos inesperados podem surgir. Para garantir que comportamentos imprevistos não prejudiquem o funcionamento do sistema, e assim assegurar os requisitos do software, técnicas de teste de software podem ser empregadas como uma das estratégias. 

Testes de software apesar de melhorarem a qualidade do software não tem a capacidade de eliminar todos os erros do sistema pelo simples fato de não ser possível testar completamente todo o código. Estratégias são tomadas para garantir a melhor cobertura possível do sistema com os recursos disponíveis, tempo e custo, e para isso é necessário avaliar a testabilidade do sistema, ou seja considerar quantos casos de testes são necessários para garantir que aquele software está minimamente coberto para garantir uma boa qualidade.

Assim, esta dissertação tem por objetivo propor um método para avaliar a testabilidade em SMA que empregam o modelo de organização $\mathcal{M}$oise$^{+}$, utilizando Rede de Petri (RP) como ferramenta de descrição e análise. O método é baseado em uma técnica de avaliação de testabilidade para agentes \textit{Belief-desire-intention}, que agora neste trabalho o modelo $\mathcal{M}$oise$^{+}$ do SMA deve ser mapeado para Redes de Petri para realizar a análise. O resultado indica o número de casos de testes necessários para garantir através da abordagem todos os caminhos uma boa cobertura testes.
\end{abstract}